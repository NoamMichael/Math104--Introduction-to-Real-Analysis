\documentclass[../master/master.tex]{subfiles}

\begin{document}

%----------------------------------------------------------------------
% LECTURE 9: Series; The number e
% Date: February 19, 2026
%----------------------------------------------------------------------
\renewcommand{\lecturenum}{9}
\renewcommand{\lecturedate}{February 19, 2026}
\renewcommand{\lecturetopic}{Series; The number $e$}

\section{Lecture \lecturenum : \lecturedate}

\begin{lecturesummary}
\textbf{Lecture Overview:} [High-level summary of the entire lecture. What are the main goals? What key concepts are introduced? How do they connect to the bigger picture of the course?]
\end{lecturesummary}

\subsection{Series}

Given a sequence $\{a_n\}$, we can consider the \defn{partial sums} $\{s_n\}$ where
\[
s_n = a_0 + a_1 + \cdots + a_n = \sum_{k=0}^{n} a_k.
\]
We say this series \defn{converges} to $s$ if
\[
s = \sum_{n=1}^{\infty} a_n,
\]
that is, if $s_n \to s$, i.e.\ the partial sums converge. This means that many of the same properties of sequences apply to series.

\subsection{Review: Cauchy Sequences}

Last time we discussed Cauchy sequences: a sequence $\{s_n\}$ converges if and only if for every $\eps > 0$ there exists $N \in \N$ such that for all $m, n \geq N$,
\[
|s_m - s_n| < \eps.
\]

\begin{notebox}
\textbf{Reader's Note:} The Cauchy criterion gives us a way to test for convergence \emph{without knowing the limit}. Instead of checking that terms get close to some value $s$, we check that terms get close to \emph{each other}. This is especially useful for series, where computing the exact sum is often difficult or impossible.
\end{notebox}

Without loss of generality we can assume $m > n$. Then
\[
|s_m - s_n| = \left| \sum_{k=n+1}^{m} a_k \right|.
\]

\subsection{Monotonic Sequences}

\begin{definition}
A monotonic sequence converges if and only if it is bounded.
\end{definition}

$\{s_n\}$ is \defn{monotonically increasing} if and only if $s_n \leq s_{n+1}$, i.e.\ all the terms $a_n$ are non-negative.

\begin{notebox}
\textbf{Reader's Note:} Why does $s_n \leq s_{n+1}$ mean all terms are non-negative? Since $s_{n+1} - s_n = a_{n+1}$, requiring $s_n \leq s_{n+1}$ is the same as requiring $a_{n+1} \geq 0$. If all terms were negative, the partial sums would be monotonically \emph{decreasing}, not increasing.
\end{notebox}

\begin{corollary}
If $m = n+1$ and $\sum a_n$ converges, then $\lim_{n \to \infty} a_n = 0$.
\end{corollary}

\begin{example}
The \defn{harmonic series}
\[
\sum \frac{1}{n} = \frac{1}{1} + \frac{1}{2} + \frac{1}{3} + \cdots
\]
\end{example}

\subsection{Comparison Tests for Convergence}

\begin{theorem}
Given two sequences $\{a_n\}$ and $\{c_n\}$:
\begin{enumerate}[label=(\alph*)]
    \item If $|a_n| \leq c_n$ for all $n \geq N$ and $\sum c_n$ converges, then $\sum a_n$ converges.
    \item If $a_n \geq d_n \geq 0$ for all $n \geq N$ and $\sum d_n$ diverges, then $\sum a_n$ diverges.
\end{enumerate}
\end{theorem}

\begin{proof}
\begin{enumerate}[label=(\alph*)]
    \item Since $\sum c_n$ converges, by the Cauchy criterion, for every $\eps > 0$ there exists $N_0$ such that for all $m > n \geq N_0$,
    \[
    \sum_{k=n+1}^{m} c_k < \eps.
    \]
    For $n \geq \max(N, N_0)$, we have
    \[
    \left| \sum_{k=n+1}^{m} a_k \right| \leq \sum_{k=n+1}^{m} |a_k| \leq \sum_{k=n+1}^{m} c_k < \eps.
    \]
    So $\sum a_n$ converges by the Cauchy criterion.

    \item Since $a_n \geq d_n \geq 0$ for $n \geq N$, we have $s_m - s_n = \sum_{k=n+1}^{m} a_k \geq \sum_{k=n+1}^{m} d_k$. Since $\sum d_n$ diverges, the partial sums of $d_n$ are not Cauchy, so the partial sums of $a_n$ are not Cauchy either. Thus $\sum a_n$ diverges.
\end{enumerate}
\end{proof}

\begin{theorem}[Cauchy Condensation Test]
If $a_0 \geq a_1 \geq \cdots \geq 0$, then $\sum a_n$ converges if and only if $\sum 2^k a_{2^k}$ converges.
\end{theorem}

\begin{proof}
Since $a_n$ is decreasing and non-negative, we group terms in blocks of powers of 2. For the partial sums $s_n = \sum_{k=0}^{n} a_k$, consider $s_{2^n}$:
\begin{align*}
s_{2^n} &= a_0 + a_1 + (a_2 + a_3) + (a_4 + a_5 + a_6 + a_7) + \cdots + (a_{2^{n-1}} + \cdots + a_{2^n - 1}) \\
&\leq a_0 + a_1 + 2a_2 + 4a_4 + \cdots + 2^{n-1} a_{2^{n-1}} \\
&= a_0 + \sum_{k=0}^{n-1} 2^k a_{2^k}.
\end{align*}
So if $\sum 2^k a_{2^k}$ converges, then the partial sums $s_{2^n}$ are bounded. Since $\{s_n\}$ is monotonically increasing (all terms are non-negative), $\{s_n\}$ is bounded, so $\sum a_n$ converges.

Conversely,
\begin{align*}
s_{2^n} &= a_0 + a_1 + (a_2 + a_3) + (a_4 + a_5 + a_6 + a_7) + \cdots + (a_{2^{n-1}} + \cdots + a_{2^n - 1}) \\
&\geq a_0 + a_1 + 2a_3 + 4a_7 + \cdots + 2^{n-1} a_{2^n - 1} \\
&\geq a_0 + \frac{1}{2}(2a_2 + 4a_4 + \cdots + 2^n a_{2^n}) \\
&= a_0 + \frac{1}{2} \sum_{k=1}^{n} 2^k a_{2^k}.
\end{align*}
So if $\sum a_n$ converges (i.e.\ the partial sums are bounded), then $\sum 2^k a_{2^k}$ is bounded and monotonically increasing, hence converges.
\end{proof}

\subsection{Special Series}

\begin{definition}
The \defn{geometric series} is
\[
\sum_{k=0}^{n} x^k = 1 + x + x^2 + \cdots + x^n.
\]
\end{definition}

Let $s_n = 1 + x + x^2 + \cdots + x^n$. Then
\[
x \cdot s_n = x + x^2 + x^3 + \cdots + x^{n+1}.
\]
Subtracting,
\[
x \cdot s_n - s_n = x^{n+1} - 1,
\]
so $s_n(x - 1) = x^{n+1} - 1$, and thus for $x \neq 1$,
\[
s_n = \frac{x^{n+1} - 1}{x - 1}.
\]
The geometric series converges if and only if $|x| < 1$, in which case $s_n \to \dfrac{1}{1 - x}$.

\begin{proof}
$(\Rightarrow)$ If $|x| \geq 1$, then $|x^n| \geq 1$ for all $n$, so $a_n = x^n \not\to 0$. Since the terms do not tend to zero, the series diverges.

$(\Leftarrow)$ If $|x| < 1$, we have
\[
\left| s_n - \frac{1}{1-x} \right| = \left| \frac{1 - x^{n+1}}{1 - x} - \frac{1}{1-x} \right| = \frac{|x|^{n+1}}{|1-x|}.
\]
Since $|x| < 1$, we have $|x|^{n+1} \to 0$ as $n \to \infty$, so $s_n \to \dfrac{1}{1-x}$.
\end{proof}

\begin{definition}
The \defn{$p$-series} is
\[
\sum \frac{1}{n^p}.
\]
If $p > 1$ it converges; if $p \leq 1$ it diverges.
\end{definition}

\begin{proof}
By the Cauchy condensation test, $\sum \frac{1}{n^p}$ converges if and only if
\[
\sum 2^k \cdot \frac{1}{(2^k)^p} = \sum 2^k \cdot 2^{-kp} = \sum 2^{k(1-p)}
\]
converges. This is a geometric series with ratio $r = 2^{1-p}$. It converges if and only if $|r| < 1$, i.e.\ $2^{1-p} < 1$, which holds if and only if $1 - p < 0$, i.e.\ $p > 1$.
\end{proof}

\begin{theorem}
If $p > 1$, then $\displaystyle\sum \frac{1}{n(\log n)^p}$ converges.
\end{theorem}

\begin{proof}
By the Cauchy condensation test, $\sum \frac{1}{n(\log n)^p}$ converges if and only if
\[
\sum 2^k \cdot \frac{1}{2^k (\log 2^k)^p} = \sum \frac{1}{(k \log 2)^p} = \frac{1}{(\log 2)^p} \sum \frac{1}{k^p}
\]
converges. Since $p > 1$, the $p$-series $\sum \frac{1}{k^p}$ converges, so the original series converges.
\end{proof}

\subsection{The Number $e$}

\begin{definition}
We define the number $\defn{$e$}$ by
\[
e = \sum_{n=0}^{\infty} \frac{1}{n!} = 1 + 1 + \frac{1}{2} + \frac{1}{6} + \frac{1}{24} + \cdots
\]
\end{definition}

\begin{theorem}
The series $\sum \frac{1}{n!}$ converges.
\end{theorem}

\begin{proof}
For $n \geq 1$, we have $n! \geq 2^{n-1}$, so
\[
\frac{1}{n!} \leq \frac{1}{2^{n-1}} = \frac{2}{2^n}.
\]
Since $\sum \frac{1}{2^n}$ is a convergent geometric series (with $|x| = \frac{1}{2} < 1$), the series $\sum \frac{1}{n!}$ converges by the comparison test.
\end{proof}

\begin{theorem}
$e$ is irrational.
\end{theorem}

\begin{proof}
Suppose for contradiction that $e = p/q$ for some $p, q \in \N$ with $q \geq 1$. Let
\[
s_n = \sum_{k=0}^{n} \frac{1}{k!}.
\]
Then $q!\, s_q$ is an integer (since $q!/ k!$ is an integer for $0 \leq k \leq q$), and $q!\, e = q!\, p/q$ is also an integer. Therefore
\[
q!\,(e - s_q) = q! \sum_{k=q+1}^{\infty} \frac{1}{k!}
\]
is a positive integer. But we can bound this:
\begin{align*}
q! \sum_{k=q+1}^{\infty} \frac{1}{k!} &= \frac{1}{q+1} + \frac{1}{(q+1)(q+2)} + \frac{1}{(q+1)(q+2)(q+3)} + \cdots \\
&\leq \frac{1}{q+1} + \frac{1}{(q+1)^2} + \frac{1}{(q+1)^3} + \cdots \\
&= \frac{1}{q+1} \cdot \frac{1}{1 - \frac{1}{q+1}} = \frac{1}{q} \leq 1.
\end{align*}
So $q!(e - s_q)$ is a positive integer that is at most $\frac{1}{q} \leq 1$. Since it is strictly less than 1 when $q \geq 2$ (and one checks $e$ is not an integer directly), this is a contradiction.
\end{proof}

\subsection{Root and Ratio Tests}

\begin{theorem}[Root Test]
Given $\sum a_n$, let $\alpha = \limsup_{n \to \infty} \sqrt[n]{|a_n|}$. Then:
\begin{enumerate}[label=(\alph*)]
    \item If $\alpha < 1$, the series converges.
    \item If $\alpha > 1$, the series diverges.
    \item If $\alpha = 1$, the test is inconclusive.
\end{enumerate}
\end{theorem}

\begin{proof}
\textbf{(a)} If $\alpha < 1$, choose $\beta$ with $\alpha < \beta < 1$. Since $\alpha = \limsup \sqrt[n]{|a_n|}$, there exists $N$ such that $\sqrt[n]{|a_n|} < \beta$ for all $n \geq N$, i.e.\ $|a_n| < \beta^n$. Since $\sum \beta^n$ is a convergent geometric series ($\beta < 1$), $\sum a_n$ converges by the comparison test.

\textbf{(b)} If $\alpha > 1$, then $\sqrt[n]{|a_n|} > 1$ for infinitely many $n$, so $|a_n| > 1$ for infinitely many $n$. Thus $a_n \not\to 0$, so the series diverges.

\textbf{(c)} Both $\sum \frac{1}{n}$ and $\sum \frac{1}{n^2}$ have $\alpha = 1$, but one diverges and the other converges.
\end{proof}

\begin{theorem}[Ratio Test]
Given $\sum a_n$ with $a_n \neq 0$:
\begin{enumerate}[label=(\alph*)]
    \item If $\limsup_{n \to \infty} \left|\dfrac{a_{n+1}}{a_n}\right| < 1$, the series converges.
    \item If $\left|\dfrac{a_{n+1}}{a_n}\right| \geq 1$ for all $n \geq N$, the series diverges.
    \item If $\limsup_{n \to \infty} \left|\dfrac{a_{n+1}}{a_n}\right| = 1$, the test is inconclusive.
\end{enumerate}
\end{theorem}

\begin{proof}
\textbf{(a)} If $\limsup \left|\frac{a_{n+1}}{a_n}\right| = r < 1$, choose $\beta$ with $r < \beta < 1$. There exists $N$ such that $\left|\frac{a_{n+1}}{a_n}\right| < \beta$ for all $n \geq N$. Then by induction,
\[
|a_{N+k}| < \beta^k |a_N|
\]
for all $k \geq 0$. Since $\sum \beta^k |a_N|$ is a convergent geometric series, $\sum a_n$ converges by comparison.

\textbf{(b)} If $\left|\frac{a_{n+1}}{a_n}\right| \geq 1$ for $n \geq N$, then $|a_n| \geq |a_N| > 0$ for all $n \geq N$, so $a_n \not\to 0$ and the series diverges.

\textbf{(c)} Same examples as the root test: $\sum \frac{1}{n}$ and $\sum \frac{1}{n^2}$.
\end{proof}

\begin{remark}
The root test is at least as powerful as the ratio test: whenever the ratio test gives a conclusion, so does the root test. This is because
\[
\limsup \left|\frac{a_{n+1}}{a_n}\right| \geq \limsup \sqrt[n]{|a_n|}.
\]
However, there are series where the root test succeeds but the ratio test is inconclusive.
\end{remark}

\end{document}
